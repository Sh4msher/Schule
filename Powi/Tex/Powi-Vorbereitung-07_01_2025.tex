\documentclass[a4paper,10pt]{article}
\usepackage[utf8]{inputenc}
\usepackage[ngerman]{babel}
\usepackage{hyperref}
\usepackage{enumitem}
\usepackage[top=2.5cm, left=2.5cm, right=2.5cm, bottom=2.5cm]{geometry}


\setlist{nolistsep}

\title{Zusammenfassung: Gesetzgebung und Lobbyismus}
\author{Shamsher Singh Kalsi}
%\author{Q1, Powi}


\begin{document}
\maketitle

\section{Gesetzgebung}

\subsection{Bundesgesetze}
Bundesgesetze gelten für alle Menschen in Deutschland. Sie werden vom Deutschen Bundestag beschlossen und sind im gesamten Bundesgebiet verbindlich.
\textbf{Wichtige Merkmale:}
\begin{itemize}
    \item Übergeordnet gegenüber Landesgesetzen.
    \item Müssen den föderalen Rahmen einhalten.
\end{itemize}

\subsection{Landesgesetze}
Landesgesetze gelten nur innerhalb eines Bundeslandes. Sie werden von den Parlamenten der jeweiligen Bundesländer (z. B. Landtage) erlassen.

\subsection{Entstehung neuer Bundesgesetze}
Vorschläge für neue Bundesgesetze können von verschiedenen Akteuren eingebracht werden. Dazu gehören:
\begin{itemize}
    \item Gruppen von Bundestagsabgeordneten, z. B. Fraktionen.
    \item Der Bundesrat (Vertretung der Länder).
    \item Die Bundesregierung, die die meisten Gesetzentwürfe entwickelt.
\end{itemize}

\subsubsection{Ablauf der Gesetzgebung}
Jeder Gesetzentwurf durchläuft in der Regel drei Lesungen im Bundestag:
\begin{enumerate}
    \item \textbf{Erste Lesung:} Diskussion über die politische Bedeutung und Ziele des Gesetzentwurfs sowie Weiterleitung an die zuständigen Fachausschüsse.
    \item \textbf{Zweite Lesung:} Beratung der Beschlussempfehlung im Plenum mit Möglichkeit für Änderungsanträge.
    \item \textbf{Dritte Lesung:} Änderungen nur an neu eingeführten Bestimmungen, gefolgt von der Schlussabstimmung.
\end{enumerate}

\subsubsection{Bundesrat und Vermittlungsausschuss}
Nach der Schlussabstimmung im Bundestag wird der Gesetzentwurf dem Bundesrat vorgelegt, der folgende Optionen hat:
\begin{itemize}
    \item Zustimmung,
    \item Einspruch,
    \item Ablehnung.
\end{itemize}

\textbf{Arten von Gesetzen:}
\begin{itemize}
    \item Zustimmungsgesetze: Erfordern die Zustimmung des Bundesrats.
    \item Einspruchsgesetze: Einspruch des Bundesrats kann vom Bundestag überstimmt werden.
\end{itemize}

Bei Uneinigkeit wird ein Kompromiss im Vermittlungsausschuss gesucht, der aus Vertretern des Bundestags und Bundesrats besteht.

\subsubsection{Unterzeichnung und Verkündung}
Nach Zustimmung wird eine Urschrift des Gesetzes erstellt. Es wird von der Bundesregierung und dem Bundespräsidenten unterschrieben und tritt nach Veröffentlichung im Bundesgesetzblatt in Kraft.
\clearpage

\section{Lobbyismus}

\subsection{Definition}
Lobbyismus bezeichnet den Versuch von Interessengruppen (z. B. Unternehmen, Verbänden, NGOs), politische Entscheidungen zu beeinflussen.

\subsection{Ziel des Lobbyismus}
Lobbyismus verfolgt mehrere Ziele:
\begin{itemize}
    \item Einflussnahme auf die Gesetzgebung zur Förderung eigener Interessen.
    \item Bereitstellung von Fachwissen und Informationen für Politiker.
    \item Sicherstellung, dass politische Entscheidungen wirtschaftliche, gesellschaftliche oder ökologische Interessen berücksichtigen.
\end{itemize}

\subsection{Akteure des Lobbyismus}
Zu den Akteuren des Lobbyismus zählen:
\begin{itemize}
    \item Wirtschaftsverbände (z. B. BDI, DIHK).
    \item Gewerkschaften (z. B. DGB, ver.di).
    \item Nichtregierungsorganisationen (NGOs, z. B. Greenpeace, Amnesty International).
    \item Beratungsunternehmen und Kanzleien (z. B. Public-Affairs-Agenturen).
\end{itemize}

\subsection{Beeinflussung von Gesetzen durch Lobbyismus}
Lobbyismus kann auf verschiedene Weise Einfluss nehmen:
\begin{itemize}
    \item Direkter Kontakt mit Abgeordneten, Ministerien oder Ausschüssen.
    \item Bereitstellung von Gutachten und Studien.
    \item Spenden und Sponsoring.
    \item Medienkampagnen zur Förderung von Themen.
\end{itemize}

\subsection{Vorteile und Gefahren des Lobbyismus}
\textbf{Vorteile:}
\begin{itemize}
    \item Politiker erhalten fachliche Expertise.
    \item Interessengruppen können Anliegen der Gesellschaft einbringen.
\end{itemize}

\textbf{Gefahren:}
\begin{itemize}
    \item Übermäßiger Einfluss von finanzstarken Akteuren.
    \item Gefahr der Korruption oder einseitiger Entscheidungen.
    \item Mangelnde Transparenz.
\end{itemize}

\subsection{Transparenz im Lobbyismus}
Um die Transparenz im Lobbyismus zu erhöhen, wurden Maßnahmen wie das Lobbyregister eingeführt. Dieses verpflichtet Lobbyorganisationen zur Registrierung und Offenlegung ihrer Tätigkeiten. Zudem gelten strengere Regeln für Abgeordnete, um Interessenskonflikte zu vermeiden.

\section*{Zusätzliche Hinweise zur Einflussnahme}
Formale und informale Einflussnahmen sind wichtige Mechanismen im politischen Prozess:
\begin{itemize}
    \item Formale Einflussnahme: Anliegen können über öffentliche Kanäle wie Stellungnahmen und Lesungen eingebracht werden.
    \item Informale Einflussnahme: Direkte Gespräche, Gutachten oder Netzwerkarbeit bieten oft größere Chancen auf Einfluss.
\end{itemize}

\end{document}
